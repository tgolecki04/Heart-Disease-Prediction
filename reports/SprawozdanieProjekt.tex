\documentclass[a4paper,12pt]{article}

\usepackage[utf8]{inputenc}
\usepackage[T1]{fontenc}
\usepackage[polish]{babel}
\usepackage{geometry}
\usepackage{graphicx}
\usepackage{hyperref}
\usepackage{amsmath, amssymb}
\usepackage{listings}
\usepackage{caption}
\usepackage{float}

\geometry{
	a4paper,
	left=25mm,
	right=25mm,
	top=25mm,
	bottom=25mm
}

\lstset{
	basicstyle=\ttfamily\footnotesize,
	breaklines=true,
	frame=single,
	captionpos=b
}

\title{Analiza danych i modelowanie klasyfikacyjne ryzyka zawału serca}
\author{tom}
\date{01.01.2026}

\begin{document}
	
	\maketitle
	\tableofcontents
	\newpage
	
	\section{Wstęp}
	Analiza zbioru danych \textit{Framingham Heart Study} pozwoliła na prześledzenie, w jaki sposób różne czynniki zdrowotne oraz styl życia wpływają bezpośrednio lub pośrednio na ryzyko wystąpienia choroby sercowo-naczyniowej w perspektywie 10 najbliższych lat. Zastosowane metody statystyczne i modele uczenia maszynowego pozwoliły na ocenę znaczenia poszczególnych zmiennych oraz sprawdzenie, które z nich mają największy wpływ na ryzyko zachorowania. W ten sposób uzyskano istotne informacje na temat głównych czynników powodujących wystąpienie choroby oraz wskazówki dotyczące środków profilaktyki, które mogą w przyszłości zmniejszyć ryzyko zawału serca.
	
	Celem projektu jest zbadanie czynników wpływających na ryzyko wystąpienia choroby wieńcowej w ciągu 10 lat wśród pacjentów. Analiza może pomóc w określeniu głównych predyktorów chorób układu sercowego oraz w oszacowaniu ryzyka dla różnych osób na podstawie ich cech demograficznych, stylu życia, nawyków i pomiarów medycznych. Wyniki analizy mogą mieć istotne znaczenie naukowe oraz być cennym źródłem profilaktyki zdrowotnej.
	
	Głównym założeniem projektu jest stworzenie nieliniowych modeli predykcyjnych zdolnych do skutecznego przewidywania potencjalnego zawału serca w najbliższych 10 latach na podstawie czynników takich jak płeć, wiek, palenie, poziom glukozy, przyjmowane leki oraz poziom cholesterolu. Projekt zakłada stworzenie minimum dwóch modeli predykcyjnych: pierwszy oparty na sieciach neuronowych (Neural Networks), a drugi na Gradient Boosting. Następnie zostanie przeprowadzone porównanie ich właściwości i statystyk oraz wybór najlepszego z nich w celu stworzenia predyktora omawianego zagrożenia. 
	
	Użytkownik będzie mógł wypełnić ankietę, a uzupełnione dane zostaną wprowadzone do modelu, który poinformuje o potencjalnej grupie ryzyka. Dodatkowo przeprowadzono podstawowe testy statystyczne oraz stworzone zostały modele poznane w trakcie studiów, takie jak regresja logistyczna czy testy różnicy wartości średniej w różnych grupach.
	
	\section{Opis projektu}
	W tej części projektu przedstawiono najważniejsze modele uczenia maszynowego wykorzystywane w praktyce do rozwiązywania problemów klasyfikacyjnych.
	
	\subsection{Regresja logistyczna}
	Regresja logistyczna to jedno z najprostszych, a jednocześnie bardzo skutecznych narzędzi do przewidywania zmiennych binarnych. Pozwala ocenić wpływ poszczególnych zmiennych na prawdopodobieństwo wystąpienia zdarzenia, jakim jest zawał serca.
	
	\subsection{Drzewa decyzyjne i metody zespołowe}
	Drzewo decyzyjne umożliwia intuicyjne podejmowanie decyzji na podstawie hierarchii warunków. Jego rozwinięciem są metody zespołowe, takie jak:
	\begin{itemize}
		\item \textbf{Random Forest} – łączy wiele drzew w jeden stabilny model, zmniejszając wariancję i poprawiając dokładność predykcji.
		\item \textbf{XGBoost} – buduje drzewa sekwencyjnie, stopniowo korygując błędy poprzednich, osiągając jedne z najlepszych wyników w zastosowaniach drzew decyzyjnych.
	\end{itemize}
	
	\subsection{Sieci neuronowe}
	Sieci neuronowe, inspirowane działaniem ludzkiego mózgu, dzięki warstwowej architekturze potrafią uchwycić bardzo złożone zależności w danych i są podstawą współczesnych sukcesów w dziedzinach takich jak rozpoznawanie obrazów czy przetwarzanie języka w czasie rzeczywistym.
	
	\subsection{Cele funkcjonalne}
	\begin{itemize}
		\item Ocena wpływu różnych czynników zdrowotnych i stylu życia na ryzyko zawału serca.
		\item Budowa modeli predykcyjnych dla ryzyka wystąpienia choroby sercowo-naczyniowej.
		\item Porównanie skuteczności różnych modeli uczenia maszynowego.
		\item Udostępnienie narzędzia umożliwiającego użytkownikowi oszacowanie indywidualnego ryzyka.
	\end{itemize}
	
	\section{Implementacja}
	Projekt został zrealizowany w Pythonie z wykorzystaniem bibliotek Pandas, NumPy, Scikit-learn, XGBoost, TensorFlow/Keras oraz PyTorch.
	
	\subsection{Struktura projektu}
	% TU_UZUPELNIJ - przykładowo:
	% data/ - dane wejściowe
	% notebooks/ - notatniki Jupyter
	% src/ - kod źródłowy
	% results/ - wyniki eksperymentów
	
	\subsection{Najważniejsze fragmenty kodu}
	\begin{lstlisting}[language=Python]
		# TU_UZUPELNIJ - wczytywanie danych, preprocessing
		import pandas as pd
		data = pd.read_csv('data/heart_data.csv')
		
		# TU_UZUPELNIJ - trening modeli
	\end{lstlisting}
	
	\section{Testy i wyniki}
	% TU_UZUPELNIJ - tabele porównawcze accuracy, F1-score, AUC dla Neural Networks, XGBoost, Random Forest
	
	\subsection{Analiza wyników}
	% TU_UZUPELNIJ - interpretacja wyników, wpływ SMOTE, znaczenie poszczególnych cech
	
	\section{Wnioski}
	% TU_UZUPELNIJ - podsumowanie projektu, trudności, możliwe usprawnienia, wskazówki profilaktyczne
	
	\section{Bibliografia}
	% TU_UZUPELNIJ - np. dokumentacja bibliotek, książki, artykuły, tutoriale
	% Przykład:
	% \begin{itemize}
		%     \item Géron, Aurélien. "Hands-On Machine Learning with Scikit-Learn, Keras & TensorFlow." O'Reilly Media, 2019.
		%     \item \url{https://xgboost.readthedocs.io/}
		% \end{itemize}
	
\end{document}
